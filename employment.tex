\begin{rubric}{实习经历}
	\entry*[2024/10 -- 2025/8]%
	\textbf{软件开发与产品应用实习生,} 上海立芯软件科技有限公司
	\par \textbf{工作内容:} 参与国产EDA工具在SMIC工艺场景下XPU,GPU等多个芯片模块的国产化替代与调优工作,负责PreCTS部分的国产EDA工具的使用流程搭建与PPA结果调优。

	\entry*[2023/10 -- 2024/9]%
	\textbf{软件开发实习生,} 上海立芯软件科技有限公司
	\par \textbf{工作内容:} 基于国产EDA底座平台,开发基于布局结果的时序预测工具,实现时序预测与逻辑综合的联动。相关算法可以基于布局结果预测时序并重新综合,提升网表与布局质量。

	\entry*[2021/7 -- 2023/9]%
	\textbf{软件开发实习生,} 上海立芯软件科技有限公司
	\par \textbf{工作内容:} 使用分析方法在布局中优化布线线长并避免时序违例,开发相关算法并应用于国产EDA工具中。在工业用例上可以取得较好的PPA结果。
	\par \textbf{论文转化:} \emph{Electrostatics-based analytical global placement for timing optimization} \cite{lin2024electrostatics}; \emph{An Analytical Placement Algorithm with Routing topology Optimization} \cite{wei2024analytical}

	\entry*[2020/9 -- 2021/6]%
	\textbf{软件开发实习生,} 上海复旦微电子集团股份有限公司
	\par \textbf{工作内容:} 参与国产FPGA配套EDA工具的开放工作,协助开发FPGA详细布局工具,相关算法可以有效提升FPGA布局的时序质量。
	%
\end{rubric}
